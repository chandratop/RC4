\documentclass{llncs}

\usepackage{amssymb}

\usepackage{amsmath}

\usepackage{url,color}



\begin{document}

\title{The Index $j$ in RC4 is not Pseudo-random}

\author{}

\institute{}

\maketitle



\begin{abstract}

In this paper we provide several theoretical evidences that the pseudo-random index $j$ of RC4 is indeed not pseudo-random. 

First we show that in long term $\Pr(j = i+1) = \frac{1}{N} - \frac{1}{N^2}$, instead 

of the random association $\frac{1}{N}$ and this happens for the non-existence of the condition $S[i] = 1 \mbox{ and } j = i+1$

that is mandatory for the non-existence of the Finney cycle. Further we also identify several results on non-existence of certain sequences 

of $j$. 

\end{abstract}

{\bf Keywords:} RC4, Non-randomness, Pseudo-random Index, Stream Cipher, Cryptography.



\section{Introduction}

As we all know, there are many results related to non-randomness of RC4 that received the attention in flagship level

cryptology conferences and journals (see for example~\cite{asia,joc,euro} and the references therein). Even after intense research 

for more than three decades on a few lines of RC4 algorithm, we are still amazed with new discoveries in this area of research. 

As we are presenting a short note, we assume that the reader is aware of RC4 algorithm. Still let us present the algorithm briefly.



In RC4, there is a $N = 256$ length array of 8-bit integers $0$ to $N-1$, that works as a permutation. There is also an $l$ length array

of bytes $K$, where $l$ may vary from 5 to 32, depending on the key length. There are also two bytes $i, j$, where $i$ is the deterministic

index that increases by 1 in each step and $j$ is updated in a manner so that it behaves pseudo-randomly.

The Key Scheduling Algorithm (KSA) of RC4 is as follows:

\begin{itemize}

\item $j = 0$; for $i = 0$ to $N-1$: $S[i] = i$; 

\item for $i = 0$ to $N-1$:

\begin{itemize}

\item[] $j = j + S[i] + K[i \bmod l]$; swap($S[i], S[j]$);

\end{itemize}

\end{itemize}

Next the pseudo-random bytes $z$ are generated during the Pseudo Random Generator Algorithm (PRGA) as follows:

\begin{itemize}

\item $i = j = 0$;

\item for $i = 0$ to $N-1$:

\begin{itemize}

\item[] $i = i+1$; $j = j + S[i]$; swap($S[i], S[j]$); $z = S[S[i]+S[j]]$; 

\end{itemize}

\end{itemize}

Note that all the additions here are modulo $N$.



\section{Non-Randomness due to non-existence of Finney cycle}

While there is long term suspicion that there could be problems with the psudo-randomness of $j$, till very recently it could not be 

observed or reported. In fact, in~\cite[Section 3.4]{joc}, non-randomness of $j$ has been studied for initial rounds and it has been

commented that the distribution of $j$ is almost uniform for higher rounds. Thus, to date, no long term pseudo-randomness 

of the index $j$ has been reported.



It has been observed by Finney~\cite{finney} that if $S[i] = 1$ and $j = i+1$, then RC4 lands into a short cycle of length $N(N-1)$.

Fortunately (or knowing this very well), the design of RC4 by Rivest

considers the initialization of RC4 PRGA as $i = j = 0$. Thus, during RC4 PGRA, the Finney cycle cannot occur, i.e., if

$\Pr(S[i] = 1)$, then $\Pr(j = i+1) = 0$. This provides

the non-randomness in $j$.



\begin{theorem}

\label{th1}

During RC4 PRGA, $\Pr(j = i+1) = \frac{1}{N} - \frac{1}{N^2}$, under certain usual assumptions.

\end{theorem} 

\begin{proof}

We have

\begin{eqnarray*}

\Pr(j = i+1) & = & \Pr(j = i+1, S[i] = 1) + \Pr(j = i+1, S[i] \neq 1)\\

             & = & 0 + \Pr(j = i+1 | S[i] \neq 1) \cdot \Pr(S[i] \neq 1)\\

             & = & \frac{1}{N} \cdot (1 - \frac{1}{N}) = \frac{1}{N} - \frac{1}{N^2}.

\end{eqnarray*} 

Here we consider $\Pr(j = i+1 | S[i] \neq 1) = \frac{1}{N}$ under usual 

randomness assumption (it has been checked by experiments too). 

Further, considering $S$ as a random permutation, we get $\Pr(S[i] \neq 1) = 1 - \frac{1}{N}$. \qed

\end{proof}



In fact, one can sharpen this result slightly by using Glimpse theorem as follows. Though it happens generally once out of $N$ rounds

during the PRGA.

\begin{corollary}

During RC4 PRGA, $\Pr(j = i+1 | i = z+1) = \frac{1}{N} - \frac{2}{N^2} + \frac{1}{N^3}$.

\end{corollary}

\begin{proof}

We refer to Glimpse theorem~\cite{jenkins} that says, $\Pr(S[j] = i - z) = \frac{2}{N} - \frac{1}{N^2}$ after the swap of

$S[i]$ and $S[j]$. Consider the situation when $S[i] = 1$ before the swap. That means $S[j] = 1$ after the swap. 

Thus, $\Pr(S[i] = 1 | i = z+1) = \frac{2}{N} - \frac{1}{N^2}$. Hence, we have the following:

\begin{eqnarray*}

\Pr(j = i+1 | i = z+1) & = & \Pr(j = i+1, S[i] = 1 | i = z+1)\\

                       &   & + \Pr(j = i+1, S[i] \neq 1 | i = z+1)\\

             & = & 0\\ 

             &   & + \Pr(j = i+1 | S[i] \neq 1, i = z+1)\\

             &   & \cdot \Pr(S[i] \neq 1 | i = z+1)\\

             & = & \frac{1}{N} \cdot (1 - \frac{2}{N} + \frac{1}{N^2}) = \frac{1}{N} - \frac{2}{N^2} + \frac{1}{N^3}.

\end{eqnarray*} 

We consider the usual assumptions as in Theorem~\ref{th1}. \qed

\end{proof}



Since we make a few assumptions, it is important to validate the results and

the experimental data indeed supports the theoretical claims mentioned above. 



\section{Non-existent sequences of $j$ over several rounds}

\begin{theorem}

\label{th2}

During RC4 PRGA, in $3$ consecutive rounds ($r$, $r+1$ and $r+2$), $j$ cannot take $3$ consecutive integer values. In other words, there is no $r$ such that $j_{r+2} = j_{r+1} + 1 = j_{r} + 2$.

\end{theorem}

\begin{proof}
Let us first consider a situation where $j$ has increased by $1$ from round $r$ to round $r+1$. \\

So $j_{r} + 1 = j_{r+1}$, which implies $S_{r}[i_{r} + 1] = S_{r+1}[j_{r+1}] = 1$. \\

For terminology, we have considered $S_{r}[k]$ as the value of the array at $k$-th index after the swap is done in round $r$. \\

In RC4 PRGA, a Finney cycle cannot happen. Hence, $i_{r+1}$ cannot take the value of $(j_{r+1} - 1)$. Hence $S_{r+1}[i_{r+1}+1]$ cannot be $1$. \\

Thus it would not be possible to have $j_{r+2} = j_{r+1} + 1$. Hence the proof. \qed

\end{proof}

As a corollary to the above theorem, we can prove the following probabilistic result.

\begin{corollary}

During RC4 PRGA, $\Pr(j_{r+2} = j_{r}+2) = \frac{1}{N} - \frac{1}{N^2}$, under certain usual assumptions.

\end{corollary} 

\begin{proof}

We have

\begin{eqnarray*}

\Pr(j_{r+2} = j_{r}+2) & = & \Pr(j_{r+2} = j_{r}+2, j_{r+1} = j_{r}+1) + \Pr(j_{r+2} = j_{r}+2, j_{r+1} \neq j_{r}+1)\\

             & = & 0 + \Pr(j_{r+2} = j_{r}+2 | j_{r+1} \neq j_{r}+1) \cdot \Pr(j_{r+1} \neq j_{r}+1)\\

             & = & $\frac{1}{N}$ \cdot (1 - \frac{1}{N})
             
             & = & $\frac{1}{N}$ - $\frac{1}{N^2}$.

\end{eqnarray*} 

Here we consider $\Pr(j_{r+2} & = & j_{r}+2 | j_{r+1} \neq j_{r}+1) = \frac{1}{N}$ under usual randomness assumption. 

Further, considering $S$ as a random permutation, we get $\Pr(j_{r+1} \neq j_{r}+1) = 1 - \frac{1}{N}$. \qed

\end{proof}

\begin{theorem}

\label{th3}

In at most three consecutive rounds $(r$, $r+1$ and $r+2)$, the value of $j$ can remain constant $(j_{r} = j_{r+1} = j_{r+2})$ or in other words there cannot exist any $r$ for which $(j_{r} = j_{r+1} = j_{r+2} = j_{r+3})$.

\end{theorem}

\begin{proof}
Let us denote the difference between the $j$ values of two consecutive rounds as $d$, i.e., $d_{r+1} = (j_{r+1} - j_{r})$. \\

\noindent Clearly, $S_{r}[i_{r}+1] = S_{r+1}[j_{r+1}] = d_{r+1}$. \\

\noindent We prove the claim now by contradiction. \\

\noindent Let us assume that it is possible to find an $r$ and a permutation $S$ such that $j_{r} = j_{r+1} = j_{r+2} = j_{r+3}$ \\

\noindent So, $d_{r+1}=d_{r+2}=d_{r+3}=0$, which implies -\\ 


\begin{enumerate}
\item $S_{r+1}[j_{r+1}] = S_{r+1}[i_{r+1}+1] = 0$ and \\
\item $S_{r+2}[i_{r+2}+1] = 0$ \\
\end{enumerate}


\noindent From 1, one can derive that \\


$j_{r+1}= i_{r+1}+1 = i_{r+2}$ \\

Since $S_{r+1}[i_{r+1}+1] = 0$, the $j$ value will not change from $(r+1)$ to $(r+2)$ round. \\

Therefore, $j_{r+2} = j_{r+1} = i_{r+2}$ and 
$S_{r+2}[j_{r+2}] = S_{r+2}[i_{r+2}] = 0$\\


\noindent Hence, $S_{r+2}[i_{r+2}+1]$ cannot be $0$ and that contradicts equation 2. Thus we prove the lemma by contradiction. \qed

\end{proof}

\begin{corollary}

If $(j_{r} = j_{r+1} = j_{r+2})$ then $i_{r+2} = j_{r+2}$ and $S_{r+1}[j_{r+1}] = S_{r+2}[i_{r+2}] = S_{r+2}[j_{r+2}] = 0$

\end{corollary} 

\begin{proof}
The result has already been proved in intermediate steps of the previous theorem.
\end{proof}

\begin{corollary}

In two consecutive rounds $(r$ and $r+1)$, if the value of $j$ remains constant $($i.e., $j_{r}=j_{r+1})$ then $S_{r+1}[j_{r+1}]$ must be $0$.

\end{corollary} 

\begin{proof}
Using the same notation as used in Corollary 3 we observe that 
$d_{r+1} = (j_{r+1} - j_{r}) = 0$\\

Therefore, $S_{r+1}[j_{r+1}] = S_r[i_{r}+1] = d_{r+1} = 0$. Hence the proof. \qed

\end{proof}

\begin{corollary}

Once a value of $j$ gets repeated in three consecutive rounds $(r$, $r+1$ and $r+2)$, no value can immediately be repeated in subsequent two rounds $($for $N>2)$. In other words, if $j_{r}=j_{r+1}=j_{r+2}$ it is not possible to have $j_{r+3}=j_{r+4}$.

\end{corollary} 

\begin{proof}
\noindent From Corollary 3, we know that \\

if $j_{r}=j_{r+1}=j_{r+2}$ then $S_{r+1}[j_{r+1}] = S_{r+2}[i_{r+2}] = S_{r+2}[j_{r+2}] = 0$ \\

\noindent So $S_{r+2}[i_{r+2}+1]$ cannot be $0$ and hence  $j_{r+3}$ must be different from  $j_{r+2}$. \\

Therefore, $S_{r+3}[i_{r+3}]$ as well as $S_{r+3}[j_{r+3}]$ would be non-zero (please note that it is possible for both $i$ and $j$ to be same in this round).\\

\noindent Next, $(i_{r+3}+1)$ cannot be same as $i_{r+2}$ for $N>2$. This implies, $S_{r+3}[i_{r+3}+1]$ cannot be $0$. \\

\noindent Thus, $j_{r+4}$ cannot be equal to $j_{r+3}$. Hence the proof. \qed

\end{proof}

\begin{theorem}

During RC4 PRGA, there cannot be a continuously decreasing sequence of $j$ of length more than $3$ or in other words there cannot exist any $r$ for which $(j_{r} - j_{r+1}) = (j_{r+1} - j_{r+2}) = (j_{r+2} - j_{r+3}) = k$ where $(k < N-1)$. 

\end{theorem} 

\begin{proof}

Let us consider an $r$ such that -\\

$(j_{r} - j_{r+1}) = (j_{r+1} - j_{r+2}) = k$ where $(k < N-1)$ \\

\noindent Since, $(j_{r} - j_{r+1}) = k$, it can be said that $S_{r}[i_{r}+1] = (N-k)$. \\

\noindent Similarly, as $(j_{r+1} - j_{r+2}) = k$, $S_{r+1}[i_{r+1}+1] = (N-k)$. \\

\noindent From the above two equations, it can be concluded that -
$i_{r+1} = j_{r+1}-1$ \\

In RC4 PRGA, since a Finney cycle cannot happen, $k$ cannot take the value of $(N-1)$ and hence $(k < N-1)$. \\

\noindent After the swap operation is completed in round $(r+2)$, 
$S_{r+2}[j_{r+2}] = (N-k)$. \\

\noindent Also, we know that $(j_{r+1} - j_{r+2}) = k$ which implies 
$j_{r+2} = j_{r+1} + (N-k)$. \\

\noindent As $(k < N-1)$, it is not possible to have $j_{r+2} = j_{r+1} + 1$. \\

So, $i_{r+2}$ cannot be one less than $j_{r+2}$ implying it would not be possible to have $(j_{r+2} - j_{r+3}) = k$. Hence the proof. \qed \\

\end{proof}

\begin{corollary}

During RC4 PRGA, there cannot be a continuously increasing sequence of $j$ of length more than $3$ or in other words there cannot exist any $r$ for which $(j_{r+1} - j_{r}) = (j_{r+2} - j_{r+1}) = (j_{r+3} - j_{r+2}) = k$ where $(k > 1)$. 
\end{corollary} 

\begin{proof}

Any increase of a $j$ value between two successive rounds by $k$ can be considered as a decrease of the $j$ value between two successive rounds by $(N-k)$. Here, $(k > 1)$. Hence, we may apply the previous theorem and arrive at this result. \qed

\end{proof}

To illustrate Theorem 4 and Corollary 6, once can say that the following sequences of $j$ can never happen in RC4 cycle (where $N=8$). \\

For example, increasing sequences like $(1, 3, 5, 7)$ or $(1, 4, 7, 2)$ or $(1, 5, 1, 5)$ etc. and decreasing sequences like $(7, 6, 5, 4)$ or $(7, 5, 3, 1)$ or $(7, 4, 1, 6)$ etc. can never happen in RC4. \\

\begin{theorem}

In RC4, there must be more than one non-Finney cycles. 

\end{theorem}

\begin{proof}

We prove the above claim by contradiction. \\

Let us assume that for some value of $N$ (of the form $2^n$) there exists only $1$ non-Finney cycle. \\

Total number of permutations (of $S$-Box of length $N$ together with $i$ and $j$) is $(N!).(N^2)$. \\

Hence, by eliminating the permutations pertaining to Finney cycle, we get that the number of permutations as part of the non-Finney cycle would be \\
 
$T = (N!).(N^2) - (N!)$ \\

We now use the results derived by Mister \& Tavares(~\cite{mister}). \\

Let us denote by $S_{0}$, any permutation of $S$-Box where $i=0$ and $j=0$. Since there is only $1$ non-Finney cycle, all possible $(N-1)$ right shifts of $S_{0}$ must be part of the same cycle. Let the right-shifts appear in the cycle as per the sequence $d_{1}, d_{2},..., d_{N-1}$ where $d_{0} (=d_{N})$ is understood to be same $0$ (no-shift). As per the Cycle Partitioning Theorem (~\cite{mister}) one can say that if $L$ is the distance (number of rounds required to reach $d_{i+1}$ from $d_{i}$), then \\

$L = \frac{T}{N} = (N-1)!.(N^2) - (N-1)!$ \\

Let us denote $d_{1}$ by $k$ (where $1<=k<=(N-1)$). Also, let $S_{1}$ be the permutation of $S$-box corresponding to the $k$-shift.\\

$S_{0}$ starts with $i=0$, hence after $k$ shifts $i=k$. So $(L-k)$ must be divisible by $N$ - since $k$ steps before $S_{1}$, $i$ must be equal to $0$.\\

Now $(L-k)$ is same as $[(T/N) - k]$ which equals $(N-1)!.(N^2) - (N-1)! - k$ \\

Since $N$ is of the form $2^n$, the first two terms of the above expression would be divisible by $N$. But the third term would not be divisible by $N$.\\ 

Hence we reach a contradiction. This implies our initial assumption is incorrect. \\

Therefore, in RC4, there must be more than $1$ non-Finney cycles. \qed

\end{proof}

We now prove the following theorem by extending the Cycle Partitioning results of Mister \& Tavares(~\cite{mister}).

\begin{theorem}
Let $S_{0}$ be the initial permutation of $S$-Box $(i=0$, $j=0)$ in an RC4 cycle with right-shifts that appear as per sequence of $(d_{1}, d_{2},..., d_{k-1})$ where $d_{0} (=d_{k})$ represents the original permutation or in other words a complete rotation by $N$-bytes. In that case,
\begin{enumerate}
    \item \hspace $k = \frac{LCM(N, d_1)}{d_1}$ 
    \item \hspace $k$ must be of the form $2^m$ where possible values for $m$ could be $(1,2,...,N)$
    \item \hspace If $T$ is the length of the cycle that starts with $(S_{0}, i=0, j=0)$, and $(m>0)$, then there must be at least $\frac{N}{2^m}=2^(n-m)$ disjoint cycles in the state space of cycle length $T$
    \item \hspace $(\frac{T}{k}-d_1)$ must be divisible by $N$ apart from the condition that $T$ must be divisible by $N$
\end{enumerate}
\end{theorem}

\begin{proof}
The claim for each part of the theorem is proved below -\\
\begin{enumerate}
    \item \hspace As $d_1$ is the first shifted occurrence in the cycle, it is clear that after repeating the same same shift $k$ number of times it must come back to the original permutation. Hence, $k$ times $d_1$ must be a multiple of $N$. Again the multiple must be the least one as otherwise the permutation would have come back to the original state even before the length of $T$ which is not possible. Therefore, the product of $k$ and $d_1$ must be same as $LCM(N, d_1)$. Hence, the result.
    \item \hspace Since $d_1$ times $k$ is of the form $LCM(N, d_1)$ and $N$ is of the form $2^n$, $k$ must be of the form $2^m$ where possible values for $m$ could be $(1,2,...,N)$ \\
    \item \hspace Since $k$ is of the form $2^m$ where $(m>0)$, there would be $2^(n-m)$ disjoint cycles in the state space of cycle length $T$. Please note that it is possible that some of these cycles may never appear in real RC4 cycles as they may not contain any permutation corresponding to $(i=0, j=0)$. The reason of qualifying the result with "at least" is to highlight that there may be other cycles in RC4 state space that somehow have the same length as the current cycle under consideration. 
    \item \hspace When the permutation reaches the first right shift of $d_1$, value of index $i$ would also be $d_1$. At this point, the number of rounds that would have occurred is $\frac{T}{k}$ and $d_1$ rounds earlier to this point the value of $i$ must have been $0$. \\
    Hence, $(\frac{T}{k}-d_1)$ must be divisible by $N$ apart from the condition that $T$ must be divisible by $N$
\end{enumerate}
\end{proof}

If we assume that the RC4 state traversal is equivalent to a random traversal (although this is not strictly true - since not all states can be reached in real RC4 starting with the condition of $i=0$ and $j=0$), the chance of $S_0$ arriving first at right shifted permutation of $1$ byte in RC4 cycle, is only $1$ out of $N$. This implies, the state space of RC4 is more likely to consist of multiple cycles of same length than a few cycles of unique lengths.

\section{Conclusion}

Rewrite



The pseudo-randomness of the index $j$ in RC4 has been an open question for quite some time. In this note we show that

$j$ is indeed not pseudo-random in long term evolution of RC4 PRGA where we consider $S$ as a pseudo-random permutation. 

To the best of our knowledge, this result has not been noted earlier. The implication of this result could be interesting 

to obtain further non-randomness in the evolution of RC4. Moreover, the result may be utilized to obtain additional biases 

at the initial stage of RC4 PRGA where the permutation $S$ has certain non-randomness.  



\begin{thebibliography}{10}



\bibitem{finney}

H. Finney. 

An RC4 cycle that can't happen. Post in sci.crypt, September 1994.



\bibitem{jenkins}

R. J. Jenkins. ISAAC and RC4. 1996. 

Available at \url{http://burtleburtle.net/bob/rand/isaac.html} 

[last accessed on October 25, 2015].



\bibitem{asia} 

K. G. Paterson, B. Poettering and J. C. N. Schuldt. 

Big Bias Hunting in Amazonia: Large-scale Computation and Exploitation of RC4 Biases. ASIACRYPT 2014. 

LNCS, Part 1, pp. 398--419, Vol. 8873, 2014.



\bibitem{joc} 

S. SenGupta, S. Maitra, G. Paul, S. Sarkar.

(Non--)Random Sequences from (Non--)Random Permutations -- Analysis of RC4 stream cipher.

Journal of Cryptology, 27(1):67--108, 2014

\bibitem{euro}

P. Sepehrdad, S. Vaudenay, and M. Vuagnoux. 

Statistical Attack on RC4 - Distinguishing WPA.

EUROCRYPT 2011. LNCS pp. 343--363, Vol. 6632, 2011.

\bibitem{mister} 

S. Mister and S. E. Tavares. 

Cryptanalysis of RC4-like Ciphers. International Workshop on Selected Areas in Cryptography. 

SAC 1998, pp. 131-143, 1998.

\end{thebibliography}

\end{document}
