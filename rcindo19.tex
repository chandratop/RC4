\documentclass{llncs}

\usepackage{amssymb}

\usepackage{amsmath}

\usepackage{url,color}



\begin{document}

\title{The Index $j$ in RC4 is not Pseudo-random}

\author{}

\institute{}

\maketitle



\begin{abstract}

In this paper we provide several theoretical evidences that the pseudo-random index $j$ of RC4 is indeed not pseudo-random. 

First we show that in long term $\Pr(j = i+1) = \frac{1}{N} - \frac{1}{N^2}$, instead 

of the random association $\frac{1}{N}$ and this happens for the non-existence of the condition $S[i] = 1 \mbox{ and } j = i+1$

that is mandatory for the non-existence of the Finney cycle. Further we also identify several results on non-existence of certain sequences 

of $j$. 

\end{abstract}

{\bf Keywords:} RC4, Non-randomness, Pseudo-random Index, Stream Cipher, Cryptography.



\section{Introduction}

As we all know, there are many results related to non-randomness of RC4 that received the attention in flagship level

cryptology conferences and journals (see for example~\cite{asia,joc,euro} and the references therein). Even after intense research 

for more than three decades on a few lines of RC4 algorithm, we are still amazed with new discoveries in this area of research. 

As we are presenting a short note, we assume that the reader is aware of RC4 algorithm. Still let us present the algorithm briefly.



In RC4, there is a $N = 256$ length array of 8-bit integers $0$ to $N-1$, that works as a permutation. There is also an $l$ length array

of bytes $K$, where $l$ may vary from 5 to 32, depending on the key length. There are also two bytes $i, j$, where $i$ is the deterministic

index that increases by 1 in each step and $j$ is updated in a manner so that it behaves pseudo-randomly.

The Key Scheduling Algorithm (KSA) of RC4 is as follows:

\begin{itemize}

\item $j = 0$; for $i = 0$ to $N-1$: $S[i] = i$; 

\item for $i = 0$ to $N-1$:

\begin{itemize}

\item[] $j = j + S[i] + K[i \bmod l]$; swap($S[i], S[j]$);

\end{itemize}

\end{itemize}

Next the pseudo-random bytes $z$ are generated during the Pseudo Random Generator Algorithm (PRGA) as follows:

\begin{itemize}

\item $i = j = 0$;

\item for $i = 0$ to $N-1$:

\begin{itemize}

\item[] $i = i+1$; $j = j + S[i]$; swap($S[i], S[j]$); $z = S[S[i]+S[j]]$; 

\end{itemize}

\end{itemize}

Note that all the additions here are modulo $N$.



\section{Non-Randomness due to non-existence of Finney cycle}

While there is long term suspicion that there could be problems with the psudo-randomness of $j$, till very recently it could not be 

observed or reported. In fact, in~\cite[Section 3.4]{joc}, non-randomness of $j$ has been studied for initial rounds and it has been

commented that the distribution of $j$ is almost uniform for higher rounds. Thus, to date, no long term pseudo-randomness 

of the index $j$ has been reported.



It has been observed by Finney~\cite{finney} that if $S[i] = 1$ and $j = i+1$, then RC4 lands into a short cycle of length $N(N-1)$.

Fortunately (or knowing this very well), the design of RC4 by Rivest

considers the initialization of RC4 PRGA as $i = j = 0$. Thus, during RC4 PGRA, the Finney cycle cannot occur, i.e., if

$\Pr(S[i] = 1)$, then $\Pr(j = i+1) = 0$. This provides

the non-randomness in $j$.



\begin{theorem}

\label{th1}

During RC4 PRGA, $\Pr(j = i+1) = \frac{1}{N} - \frac{1}{N^2}$, under certain usual assumptions.

\end{theorem} 

\begin{proof}

We have

\begin{eqnarray*}

\Pr(j = i+1) & = & \Pr(j = i+1, S[i] = 1) + \Pr(j = i+1, S[i] \neq 1)\\

             & = & 0 + \Pr(j = i+1 | S[i] \neq 1) \cdot \Pr(S[i] \neq 1)\\

             & = & \frac{1}{N} \cdot (1 - \frac{1}{N}) = \frac{1}{N} - \frac{1}{N^2}.

\end{eqnarray*} 

Here we consider $\Pr(j = i+1 | S[i] \neq 1) = \frac{1}{N}$ under usual 

randomness assumption (it has been checked by experiments too). 

Further, considering $S$ as a random permutation, we get $\Pr(S[i] \neq 1) = 1 - \frac{1}{N}$. \qed

\end{proof}



In fact, one can sharpen this result slightly by using Glimpse theorem as follows. Though it happens generally once out of $N$ rounds

during the PRGA.

\begin{corollary}

During RC4 PRGA, $\Pr(j = i+1 | i = z+1) = \frac{1}{N} - \frac{2}{N^2} + \frac{1}{N^3}$.

\end{corollary}

\begin{proof}

We refer to Glimpse theorem~\cite{jenkins} that says, $\Pr(S[j] = i - z) = \frac{2}{N} - \frac{1}{N^2}$ after the swap of

$S[i]$ and $S[j]$. Consider the situation when $S[i] = 1$ before the swap. That means $S[j] = 1$ after the swap. 

Thus, $\Pr(S[i] = 1 | i = z+1) = \frac{2}{N} - \frac{1}{N^2}$. Hence, we have the following:

\begin{eqnarray*}

\Pr(j = i+1 | i = z+1) & = & \Pr(j = i+1, S[i] = 1 | i = z+1)\\

                       &   & + \Pr(j = i+1, S[i] \neq 1 | i = z+1)\\

             & = & 0\\ 

             &   & + \Pr(j = i+1 | S[i] \neq 1, i = z+1)\\

             &   & \cdot \Pr(S[i] \neq 1 | i = z+1)\\

             & = & \frac{1}{N} \cdot (1 - \frac{2}{N} + \frac{1}{N^2}) = \frac{1}{N} - \frac{2}{N^2} + \frac{1}{N^3}.

\end{eqnarray*} 

We consider the usual assumptions as in Theorem~\ref{th1}. \qed

\end{proof}



Since we make a few assumptions, it is important to validate the results and

the experimental data indeed supports the theoretical claims mentioned above. 



\section{Non-existant sequences of $j$ over several rounds}



\section{Conclusion}

Rewrite



The pseudo-randomness of the index $j$ in RC4 has been an open question for quite some time. In this note we show that

$j$ is indeed not pseudo-random in long term evolution of RC4 PRGA where we consider $S$ as a pseudo-random permutation. 

To the best of our knowledge, this result has not been noted earlier. The implication of this result could be interesting 

to obtain further non-randomness in the evolution of RC4. Moreover, the result may be utilized to obtain additional biases 

at the initial stage of RC4 PRGA where the permutation $S$ has certain non-randomness.  



\begin{thebibliography}{10}



\bibitem{finney}

H. Finney. 

An RC4 cycle that can't happen. Post in sci.crypt, September 1994.



\bibitem{jenkins}

R. J. Jenkins. ISAAC and RC4. 1996. 

Available at \url{http://burtleburtle.net/bob/rand/isaac.html} 

[last accessed on October 25, 2015].



\bibitem{asia} 

K. G. Paterson, B. Poettering and J. C. N. Schuldt. 

Big Bias Hunting in Amazonia: Large-scale Computation and Exploitation of RC4 Biases. ASIACRYPT 2014. 

LNCS, Part 1, pp. 398--419, Vol. 8873, 2014.



\bibitem{joc} 

S. SenGupta, S. Maitra, G. Paul, S. Sarkar.

(Non--)Random Sequences from (Non--)Random Permutations -- Analysis of RC4 stream cipher.

Journal of Cryptology, 27(1):67--108, 2014



\bibitem{euro}

P. Sepehrdad, S. Vaudenay, and M. Vuagnoux. 

Statistical Attack on RC4 - Distinguishing WPA.

EUROCRYPT 2011. LNCS pp. 343--363, Vol. 6632, 2011.



\end{thebibliography}

\end{document}

